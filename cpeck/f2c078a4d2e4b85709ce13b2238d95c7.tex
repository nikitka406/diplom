\documentclass[a5paper,10pt]{extreport}
\usepackage[english,russian]{babel}
\usepackage[utf8x]{inputenc}
\usepackage{latexsym,mathrsfs}
\usepackage{stmaryrd, enumitem}
\usepackage{amsthm,amsfonts,amssymb,amsmath}
\usepackage{geometry}
\usepackage{tempora}
\usepackage[pdftex]{graphicx}

\geometry{top=17mm} \geometry{bottom=20mm}
\geometry{left=17mm} \geometry{right=17mm}
\linespread{1.1}
\parindent=5mm

\def\udcK#1{\noindent УДК~{#1}}
\def\titleK#1{\begin{center}{\textbf {#1}}\end{center}}
\def\authorK#1{\begin{center}{#1}\end{center}}
\def\advisorK#1{Научный руководитель — {#1}}

\newenvironment{abstractK}{}{~\newline\parindent=5mm\rule{5.5cm}{0.3pt}}
\newenvironment{bibliographyK}{\footnotesize \begin{enumerate}[label={[\arabic*]}]}{\end{enumerate}}

\newtheorem{lemma}{Лемма}
\newtheorem{theorem}{Теорема}
\newtheorem{corollary}{Следствие}
\newtheorem{proposition}{Предложение}
\theoremstyle{definition}
\newtheorem{definition}{Определение}
\theoremstyle{definition}
\newtheorem{question}{Вопрос}
\theoremstyle{definition}
\newtheorem{conjecture}{Гипотеза}

%ВАЖНО: Не менять и не добавлять ничего выше этой строки.
%Изменения вносить только внутри окружения \begin{document}\end{document}
\begin{document}
%УДК
\udcK{519.8}
%Название доклада
\titleK{Генетический алгоритм для задачи маршрутизации буровых установок}
%Информация об авторе
\authorK{Н. А. Попов\\
Новосибирский государственный университет}
%Текст тезисов доклада
\parindent=0.5cm

\jjВ задаче маршрутизации буровых установок задано множество объектов для изыскательских работ и множество буровых установок. Известно расстояние между объектами и время на перемещение между ними. Заданы временные окна, которые определяют временной интервал проведения работ на каждом объекте. Если установка приехала на объект раньше срока, то она ждет начало окна. Требуется построить график проведения работ для каждой установки и минимизировать их суммарное перемещение между объектами. Разрешается совместное выполнение работ на объекте несколькими установками.

\parindent=0.5cm

\jjДля решения задачи разработана модель целочисленного линейного программирования, позволяющая находить оптимальные решения при малой размерности задачи. Для задач реальной размерности разработан генетический алгоритм, основанный на создании популяции решений, каждое из которых создается с помощью случайного перемещения объектов из маршрута одной установки в маршрут другой установки [1]. Далее применяются алгоритмы скрещивания и мутации. Для скрещивания решений используются три процедуры, ранее показавшие хорошие результаты [2]. Для мутаций используются алгоритмы локальной перестройки. В случае нарушения ограничений на временные окна накладывается штраф. В стартовом решении на каждую скважину отправляется одна буровая установка. Если их количество превышает заданный порог, то «арендуем» дополнительные установки и добавляем в целевую функцию стоимость аренды.

\parindent=0.5cm
Разработанный алгоритм запрограммирован на языке Python и тестировался на примерах с разным числом объектов, разбросанных случайным образом на плоскости. Приводится сравнение результатов работы программы с результатом коммерческого пакета Gurobi.

\noindent\_\_\_\_\_\_\_\_\_\_\_\_\_\_\_\_\_\_\_\_\_\_\_\_\_\_\_\_\_\_
%Ссылки на используемую литературу
\begin{bibliographyK}
\item \label{reference1}
Y. Nagata. O. Braysy. A powerful route minimization heuristic for the vehicle routing problem time windows. Oper. Res. Letters 37 (2009), 333-338
\item \label{reference2}
K. Puljic, R. Manger. Comparison of eight evolutionary crossover operators for the vehicle routing problem. Math. Commun. 18 (2013), 359–375
\end{bibliographyK}
%Научный руководитель
\advisorK{д-р физ.-мат. наук, проф. Ю. А. Кочетов}
\end{document}