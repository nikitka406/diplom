\documentclass[a5paper,10pt]{extreport}
\usepackage[english,russian]{babel}
\usepackage[utf8x]{inputenc}
\usepackage{latexsym,mathrsfs}
\usepackage{stmaryrd, enumitem}
\usepackage{amsthm,amsfonts,amssymb,amsmath}
\usepackage{geometry}
\usepackage{tempora}
\usepackage[pdftex]{graphicx}

\geometry{top=17mm} \geometry{bottom=20mm}
\geometry{left=17mm} \geometry{right=17mm}
\linespread{1.1}
\parindent=5mm

\def\udcK#1{\noindent УДК~{#1}}
\def\titleK#1{\begin{center}{\textbf {#1}}\end{center}}
\def\authorK#1{\begin{center}{#1}\end{center}}
\def\advisorK#1{Научный руководитель — {#1}}

\newenvironment{abstractK}{}{~\newline\parindent=5mm\rule{5.5cm}{0.3pt}}
\newenvironment{bibliographyK}{\footnotesize \begin{enumerate}[label={[\arabic*]}]}{\end{enumerate}}

\newtheorem{lemma}{Лемма}
\newtheorem{theorem}{Теорема}
\newtheorem{corollary}{Следствие}
\newtheorem{proposition}{Предложение}
\theoremstyle{definition}
\newtheorem{definition}{Определение}
\theoremstyle{definition}
\newtheorem{question}{Вопрос}
\theoremstyle{definition}
\newtheorem{conjecture}{Гипотеза}

%ВАЖНО: Не менять и не добавлять ничего выше этой строки.
%Изменения вносить только внутри окружения \begin{document}\end{document}
\begin{document}
	%УДК
	\udcK{519.8}
	%Название доклада
	\titleK{Вероятностный поиск с запретами для задачи маршрутизации буровых установок}
	%Информация об авторе
	\authorK{А. А. Шпак\\
		Новосибирский государственный университет}
	%Текст тезисов доклада
	\parindent=0.5cm
	
	\jjВ задаче маршрутизации буровых установок задано множество объектов проведения изыскательных работ, депо, расстояния и время перемещения между ними. Время проведения работ на каждом объекте ограничивается временным окном. Дано множество всех буровых установок и число рабочих дней, которое необходимо для выполнения всех работ на объекте одной буровой установкой. Если работы проводятся несколькими установками, то длительность работ сокращается пропорционально числу привлекаемых установок. В начальный момент времени все установки находятся в депо и в конце работы в него возвращаются. Требуется минимизировать суммарное пройденное расстояние для всех буровых установок при выполнении изыскательных работ. 
	
	\parindent=0.5cm
	
	\jjДля решения задачи разработана модель частично-целочисленного линейного программирования, с ее помощью можно находить оптимальные решения, но только при небольшом числе объектов. Для решения задач большей размерности, разработан рандомизированный алгоритм локального поиска с запретами. Начальное решение получено с помощью метода Кларка и Райта, благодаря которому строится схема с кольцевыми маршрутами. Для улучшения полученного решения используются операторы локального поиска: перемещения (Relocate operator) и 2Опт (2Opt operator). На каждой итерации алгоритм ищет наилучшие решения среди соседних решений, порождаемых каждым оператором. Далее выбирается лучший ход из полученных и сохраняется в списке запретов [1]. Также в алгоритме используются штрафные функции, которые начисляют дополнительные расходы, если буровые установки не укладываются во временные окна.
	
	\parindent=0.5cm
	Разработанный алгоритм запрограммирован на языке Python. Вычислительные эксперименты проводились на данных, генерируемых случайным образом. Приводятся результаты сравнения работы алгоритма с пакетом Gurobi.
	
	\noindent\_\_\_\_\_\_\_\_\_\_\_\_\_\_\_\_\_\_\_\_\_\_\_\_\_\_\_\_\_\_
	%Ссылки на используемую литературу
	\begin{bibliographyK}
		\item \label{reference1}
		1. Berbotto, L., García, S., Nogales, F.J. A Randomized Granular Tabu Search heuristic for the split delivery vehicle routing problem. Ann Oper Res 222, 153–173 (2014)
	\end{bibliographyK}
	%Научный руководитель
	\advisorK{д-р физ.-мат. наук, проф. Ю. А. Кочетов}
\end{document}